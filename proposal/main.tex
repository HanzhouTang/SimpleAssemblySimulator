\documentclass[12pt]{extarticle}
\usepackage[utf8]{inputenc}
\usepackage{cite}

\title{Proposal}
\author{
Hanzhou Tang \\
\texttt{hanzhout@smu.edu}
\and 
Jiutian Yu \\
\texttt{jiutiany@smu.com}
}
\date{February 2019}

\begin{document}

\maketitle
\begin{abstract}
We want to build a assembly simulator which should supports basic x86 assembly instructions.
\end{abstract}
\section{Why Java}
It's tempted to implement a assembly simulator in low level language like C or C++. However, after some consideration, we decide choose Java.f
There are several reason. 
\subsection{Java is easy for memory management}
With C++11, it has smarter pointer  to make life easier \cite{josuttis2012c++}. 
However, it's implemented in library level instead of language supporting.
Sometimes, when mistakenly mixed raw pointer and smart pointer , smart pointer may become useless.
\subsection{Java is easy to test}
There are lots of powerful java testing framework to do unit test. For example, JUnit or Groovy Spock. 
Meanwhile, because Java support proxy object, it's much easy to mock obejcts and record function invoking.
\subsection{Java is easy for package manage}
With the help of gradle \cite{muschko2014gradle}, it's very easy for us to import different libraries and do deployment. We believe it could save us lots of time and effort.
\subsection{Java is easy to integrate with REST API}
If we could finish our project well, we may want to implement an online version for all users. We could easily provide REST API with the help of Spring \cite{walls2005spring}.
\section{Our registers}
We do want to implement some basic functionality of x86 platform. So we decide to imitate 10  32 bits registers. Here is a table of our registers.
The table originally comes from \cite{kusswurm2014modern}.
\begin{table}[h!]
    \centering
    \begin{tabular}{||c | p{9cm}||} 
     \hline
     Register & Descriptions \\ [0.5ex] 
     \hline
     EAX & Accumulator. 0 to 7 can be refered as AL. 8 to 15 can be refered as AH. 0 to 15 can be refered as AX. \\ 
     \hline
     EBX & Memory pointer, base Register. 0 to 7 can be refered as BL. 8 to 15 can be refered as BH. 0 to 15 can be refered as BX. \\
     \hline
     ECX & Loop control. 0 to 7 can be refered as CL. 8 to 15 can be refered as CH. 0 to 15 can be refered as CX. \\
     \hline
     EDX & Integer multiplication, integer division. 0 to 7 can be refered as DL. 8 to 15 can be refered as DH. 0 to 15 can be refered as DX. \\
     \hline
     ESI & String instruction source pointer. \\
     \hline
     EDI & String instruction destination pointer. \\
     \hline
     ESP & Stack Pointer. \\
     \hline
     EBP & Stack frame base Pointer. \\
     \hline
     EIP & Instruction pointer register. \\
     \hline
     EFLAGS & Flag register. \\
     \hline
    \end{tabular}
    \caption{The registers we want to imitate}
    \label{table:1}
\end{table}
As you can see, we ignore all segment registers. That's because according to our design, only one process can run at one time and no context switch.
We think in this case segment information is uncessary. Maybe we're wrong, we may change our decision later.  
\section{Our instruction set}
We want to support a subset of assembly instructions on x86 platform. 
By doing some research \cite{kusswurm2014modern}, we provide a table which contains all instructions we want support.
\begin{table}[h!]
    \centering
    \begin{tabular}{||c | p{9cm}||} 
     \hline
     mov & Copy data from one place to another place. \\ [0.5ex] 
     \hline
     push & Push register, memory location or immediate value onto stack.  \\ 
     \hline
     pop & Pop the first item from stack. \\
     \hline
     add & Add two number \\
     \hline
     sub &  Subtraction \\
     \hline
     cbw & Sign-extends register AL. \\
     \hline
     cwd & Sign-extends register AX. \\
     \hline
     bswap & Reverse the bytes of a 32-bit register. \\
     \hline
     and & Logic and. \\
     \hline
     or & Logic or. \\
     \hline
     xor & Logic xor. \\
     \hline
     not & Logic not. \\
     \hline
     sal/shl & Left shift. \\
     \hline
     sar & Arithmetric right shift. \\
     \hline
     shr & Logic right shift. \\
     \hline
     cmpsb/cmpsw/cmpsd & Compare the values at location. \\
     \hline
     lodsb/lodsw/lodsd & Loads the values at location. \\
     \hline
     stosb/stosw/stosd & Save the values at register to memory. \\
     \hline
     rep/repne & Repeat a specified instruction by condition \\
     \hline
     jmp/jcc/jecxz & Unconditonal/conditional jump \\
     \hline
     call & Push content to stack then do unconditonal jump.\\
     \hline
     ret & Pop stack then do unconditonal jump.\\
     \hline
     enter & Create a stack frame for function.\\
     \hline
     leave & Remove a stack frame of function.\\
     \hline
     loop/loope/loopz/loopne/loopnz & Loop.\\
     \hline
     nop & Advance the instruction pointer.\\
     \hline
    \end{tabular}
    \caption{The instructions we want to imitate}
    \label{table:2}
\end{table}

\bibliography{main}{}
\bibliographystyle{acm}


\end{document}
