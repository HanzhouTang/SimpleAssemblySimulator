\documentclass[12pt]{extarticle}
\usepackage[utf8]{inputenc}
\usepackage{cite}
\usepackage{algorithm}
\usepackage[noend]{algpseudocode}
\usepackage{listings}
\usepackage{color}

\definecolor{dkgreen}{rgb}{0,0.6,0}
\definecolor{gray}{rgb}{0.5,0.5,0.5}
\definecolor{mauve}{rgb}{0.58,0,0.82}

 
\lstdefinestyle{mystyle}{
    frame=tb,
    language=Java,
    %backgroundcolor=\color{backcolour},
    numberstyle=\tiny\color{gray},
    keywordstyle=\color{blue},
    commentstyle=\color{dkgreen},
    stringstyle=\color{mauve},
    basicstyle={\small\ttfamily},
    breakatwhitespace=true,         
    breaklines=true,                 
    captionpos=b,                    
    keepspaces=true,                 
    numbers=left,                    
    numbersep=5pt,                  
    showspaces=false,                
    showstringspaces=false,
    showtabs=false,                  
    tabsize=2
}
 
\lstset{style=mystyle}

\title{Proposal}
\author{
Hanzhou Tang \\
\texttt{hanzhout@smu.edu}
\and 
Jiutian Yu \\
\texttt{jiutiany@smu.com}
}
\date{ 2019 April}

\begin{document}


\maketitle
\begin{abstract}
We want to build an assembly simulator which should support basic x86 assembly instructions.
\end{abstract}
\section{Motivation}
To implement a basic architecture is a direct way to reflect what we are going to learn in this class. We could also say we are going to implement a simple virtual machine. Beyond simply implementing the original architecture, we want to add some new functions based on this class to complete the architecture.  
\section{Why Java}
It's tempted to implement an assembly simulator in a low level language like C or C++. However, after some considerations, we decided to choose Java.
There are several reasons which will be show below. 
\subsection{Java is easy for memory management}
With C++11, it has smarter pointers to make life easier \cite{josuttis2012c++}. 
However, it's implemented in library level instead of language supporting.
Sometimes, when we mistakenly mixed raw pointers and smart pointers , smart pointers may become useless.
\subsection{Java is easy to test}
There are lots of powerful java testing framework to do unit test. For example, JUnit or Groovy Spock. 
Meanwhile, because Java support proxy object, it's much easy to mock objects and record function invoking.
\subsection{Java is easy for package manage}
With the help of gradle \cite{muschko2014gradle}, it's very easy for us to import different libraries and do deployment. We believe it could save us lots of time and efforts.
\subsection{Java is easy to integrate with REST API}
If we could finish our project well, we may want to implement an online version for all users. We could easily provide REST API with the help of Spring \cite{walls2005spring}.
\section{Our registers}
We do want to implement some basic functionality of x86 platform. So we decide to imitate ten 32 bits registers. A table, which is Table1, of our registers is shown below.
The table originally comes from \cite{kusswurm2014modern}.

As you can see, we ignore all segment registers. The reason is that according to our design, only one process can run at one time and no context switch.
We think, in this case, segment information is unnecessary. Maybe we're wrong, we may change our decision later.  
\section{Our instruction set}
We want to support a subset of assembly instructions on x86 platform. 
By doing some researches \cite{kusswurm2014modern}, we provide a table which contains all instructions we want support.

\section{Our progress}
We finished most of our assembler and instruction set. 

In the assembler, the data segment can support expression, such as (\$ - 10)

We didn't re-invent the instruction set, instead, we follow the Intel standard to represent our instructions. For more detail, you can find at http://www.c-jump.com/CIS77/CPU/x86/lecture.html

The only thing left is to implement the virtual machine which supports our instruction. 

By some discussion, we decide to implement a 5-stage virtual machine to execute our instructions.

We also want to provide the ability to expose our internal status. We may choose to export internal status by html. If we can finish our virtual machine early.

We provide part of our implementation of assembler and instruction.

\bibliography{bibliography.bib}
\bibliographystyle{acm}
\bigskip
\begin{table}[h!]
    \centering
    \begin{tabular}{||c | p{9cm}||} 
     \hline
     Register & Descriptions \\ [0.5ex] 
     \hline
     EAX & Accumulator. 0 to 7 can be refered as AL. 8 to 15 can be refered as AH. 0 to 15 can be refered as AX. \\ 
     \hline
     EBX & Memory pointer, base Register. 0 to 7 can be refered as BL. 8 to 15 can be refered as BH. 0 to 15 can be refered as BX. \\
     \hline
     ECX & Loop control. 0 to 7 can be refered as CL. 8 to 15 can be refered as CH. 0 to 15 can be refered as CX. \\
     \hline
     EDX & Integer multiplication, integer division. 0 to 7 can be refered as DL. 8 to 15 can be refered as DH. 0 to 15 can be refered as DX. \\
     \hline
     ESI & String instruction source pointer. \\
     \hline
     EDI & String instruction destination pointer. \\
     \hline
     ESP & Stack Pointer. \\
     \hline
     EBP & Stack frame base Pointer. \\
     \hline
     EIP & Instruction pointer register. \\
     \hline
     EFLAGS & Flag register. \\
     \hline
    \end{tabular}
    \caption{The registers we want to imitate}
    \label{table:1}
\end{table}

\begin{table}[h!]
    \centering
    \begin{tabular}{||c | p{9cm}||} 
     \hline
     mov & Copy data from one place to another place. \\ [0.5ex] 
     \hline
     push & Push register, memory location or immediate value onto stack.  \\ 
     \hline
     pop & Pop the first item from stack. \\
     \hline
     add & Add two number \\
     \hline
     sub &  Subtraction \\
     \hline
     cbw & Sign-extends register AL. \\
     \hline
     cwd & Sign-extends register AX. \\
     \hline
     bswap & Reverse the bytes of a 32-bit register. \\
     \hline
     and & Logic and. \\
     \hline
     or & Logic or. \\
     \hline
     xor & Logic xor. \\
     \hline
     not & Logic not. \\
     \hline
     sal/shl & Left shift. \\
     \hline
     sar & Arithmetric right shift. \\
     \hline
     shr & Logic right shift. \\
     \hline
     cmpsb/cmpsw/cmpsd & Compare the values at location. \\
     \hline
     lodsb/lodsw/lodsd & Loads the values at location. \\
     \hline
     stosb/stosw/stosd & Save the values at register to memory. \\
     \hline
     rep/repne & Repeat a specified instruction by condition \\
     \hline
     jmp/jcc/jecxz & Unconditonal/conditional jump \\
     \hline
     call & Push content to stack then do unconditonal jump.\\
     \hline
     ret & Pop stack then do unconditonal jump.\\
     \hline
     enter & Create a stack frame for function.\\
     \hline
     leave & Remove a stack frame of function.\\
     \hline
     loop/loope/loopz/loopne/loopnz & Loop.\\
     \hline
     nop & Advance the instruction pointer.\\
     \hline
    \end{tabular}
    \caption{The instructions we want to imitate}
    \label{table:2}
\end{table}
\pagebreak

\section{Part of our source code}
\begin{enumerate}

    \item\textbf {Simpler Lexer} \\
    \begin{lstlisting}
    package assembler;
    
    import org.apache.commons.lang3.StringUtils;
    import org.apache.commons.lang3.tuple.Pair;
    import org.apache.log4j.Logger;
    import org.springframework.stereotype.Component;
    
    
    import java.math.BigInteger;
    import java.nio.file.Files;
    import java.nio.file.Paths;
    import java.util.*;
    import java.util.concurrent.atomic.AtomicInteger;
    import java.util.function.Predicate;
    import java.util.stream.Collectors;
    import java.util.stream.Stream;
    
    /**
     * The real implementation of Lexer interface.
     * The Simple will first strip all comments and then split content into different tokens by delimiters.
     * @author  Hanzhou Tang
     */
    @Component
    public class SimpleLexer implements Lexer {
        private static final Logger LOGGER = Logger.getLogger(SimpleLexer.class);
        //private Map<Integer, String> lines = null;
        private List<SourceCodeWrapper> lines = null;
        public List<SourceCodeWrapper> getLines() {
            return lines;
        }
    
        public static Set<Character> delimiters = new HashSet<>();
        public static Map<String, Token> char2Token = new HashMap<>();
    
        static {
            delimiters.add(' ');
            delimiters.add('\t');
            delimiters.add('\n');
            delimiters.add(',');
            delimiters.add(':');
            delimiters.add('+');
            delimiters.add('-');
            delimiters.add('*');
            delimiters.add('/');
            delimiters.add('[');
            delimiters.add(']');
            delimiters.add('(');
            delimiters.add(')');
            delimiters.add('$');
            delimiters.add('\'');
            char2Token.put(",", Token.Comma);
            char2Token.put(":", Token.Colon);
            char2Token.put("+", Token.Add);
            char2Token.put("-", Token.Sub);
            char2Token.put("*", Token.Mul);
            char2Token.put("/", Token.Div);
            char2Token.put("[", Token.LeftSquareBracket);
            char2Token.put("]", Token.RightSquareBracket);
            char2Token.put("\n", Token.NewLine);
            char2Token.put("(", Token.LeftParent);
            char2Token.put(")", Token.RightParent);
            char2Token.put("$", Token.DollarSign);
            char2Token.put("'",Token.Quote);
        }
    
        Status status = null;
    
        @Override
        public Status getStatus() {
            return status;
        }
    
        public void setStatus(Status status) {
            this.status = status;
        }
    
        private Pair<Integer, String> stripComments(final Map.Entry<Integer, String> record) {
            String retStr = StringUtils.chomp(record.getValue());
            int index = retStr.indexOf(';');
            if (index != -1) {
                retStr = retStr.substring(0, index);
            }
            retStr = StringUtils.strip(retStr);
            if (StringUtils.isNotEmpty(retStr)) {
                retStr += "\n";
            }
            return Pair.of(record.getKey(), retStr);
        }
    
        @Override
        public void readFile(String filename) throws Exception {
            AtomicInteger counter = new AtomicInteger();
            lines = Files.lines(Paths.get(filename))
                    .collect(Collectors.toMap(s -> counter.incrementAndGet(), s -> s))
                    .entrySet().stream().map(this::stripComments).filter(x -> StringUtils.isNotEmpty(x.getValue()))
                    .map(e -> new SourceCodeWrapper(e.getKey(),e.getValue())).collect(Collectors.toList());
            if(lines.size()>0){
                status = new Status();
                status.setCodeOfCurrentLine(lines.get(0).getCode());
            }
            else{
                throw  new Exception("file is empty");
            }
        }
    
    
        private LexemeTokenWrapper getNextLexemeAndToken(){
            Token token = getNextToken();
            if(token.equals(Token.EndofContent)){
                return new LexemeTokenWrapper(Token.EndofContent,"");
            }
            else{
                return new LexemeTokenWrapper(getStatus().getCurrentToken(),getStatus().getCurrentLexeme());
            }
    
        }
    
        @Override
        public List<LexemeTokenWrapper> lookAheadK(int k){
            Status oldStatus = new Status(getStatus());
            List<LexemeTokenWrapper> ret = Stream.generate(this::getNextLexemeAndToken).limit(k).collect(Collectors.toList());
            setStatus(oldStatus);
            return ret;
        }
    
        public static int findStr_if(final String str, final int begin, Predicate<Character> predicate) {
            int i = begin;
            for (; i < str.length(); i++) {
                if (predicate.test(str.charAt(i))) {
                    return i;
                }
            }
            return i;
        }
    
        @Override
        public Token getNextToken() {
            if (status.getIterator()>=lines.size()) {
                return Token.EndofContent;
            } else {
                if (status.getCharIndex() == status.getCodeOfCurrentLine().length()) {
                    status.setIterator(status.getIterator()+1);
                    if(status.getIterator()==lines.size()){
                        return Token.EndofContent;
                    }
                    SourceCodeWrapper wrapper = lines.get(status.getIterator());
                    status.setCharIndex(0);
                    status.setCodeOfCurrentLine(wrapper.getCode());
                    status.setLineIndex(wrapper.getLineNumber());
                    status.setCurrentToken(Token.Invalid);
                    status.setCurrentLexeme("");
                }
                status.setCurrentToken(Token.Invalid);
                final String str = status.getCodeOfCurrentLine();
                int i = findStr_if(str, status.getCharIndex(), ch -> ch == '\n' || !Character.isWhitespace(ch));
                int j = findStr_if(str, i, delimiters::contains);
                if (i == j) {
                    status.setCurrentLexeme("" + str.charAt(i));
                    status.setCharIndex(i + 1);
                    if (char2Token.containsKey(status.getCurrentLexeme())) {
                        Token token = char2Token.get(status.getCurrentLexeme());
                        status.setCurrentToken(token);
                    } else {
                        LOGGER.error("the Character (" + status.getCurrentLexeme() + ") " + "at line " + status.getLineIndex() + " is invalid");
                    }
                } else {
                    String lexeme = str.substring(i, j);
                    status.setCurrentLexeme(lexeme);
                    status.setCharIndex(j);
                    if (StringUtils.isNumeric(lexeme)) {
                        status.setCurrentToken(Token.Number);
                    } else if (lexeme.startsWith(".")) {
                        status.setCurrentToken(Token.DotString);
                    } else {
                        status.setCurrentToken(Token.String);
                    }
                }
                if (status.getCurrentToken() == Token.Invalid) {
                    LOGGER.warn("the word (" + status.getCurrentLexeme() + ") at line " + status.getLineIndex() + " is invalid");
                    //throw new Exception("the word (" + status.getCurrentLexeme() + ") at line " + status.getLineIndex() + " is invalid");
                }
                return status.getCurrentToken();
            }
        }
    
    }
    \end{lstlisting}
    
    \item\textbf {Simple Parser} \\
    \begin{lstlisting}
    package assembler;
    
    import OutputFile.DataType;
    import OutputFile.ObjFile;
    import org.apache.log4j.Logger;
    import org.springframework.beans.factory.annotation.Autowired;
    import org.springframework.stereotype.Component;
    
    import java.math.BigInteger;
    import java.util.List;
    import java.util.Optional;
    
    /**
     * A parser to convert a assembly file into binary file.
     * For now, the parser will pass code segment twice.
     * It will collect all label information in the first time.
     * In the second time, the real parsing is done.
     * @author Hanzhou Tang
     */
    
    @Component
    public class SimpleParser {
        @Autowired
        private Lexer lexer;
        private static final Logger LOGGER = Logger.getLogger(SimpleParser.class);
    
        public Lexer getLexer() {
            return lexer;
        }
    
        public ObjFile parse(String name) throws Exception {
            lexer.readFile(name);
            ObjFile objFile = new ObjFile();
            parse(objFile);
            return objFile;
        }
    
        protected void parse(ObjFile obj) throws Exception {
            LexemeTokenWrapper wrapper = lexer.lookAheadK(1).get(0);
            if (wrapper.getToken().equals(Token.DotString) && ".data".equals(wrapper.getLexeme())) {
                dataSegment(obj);
            }
        }
    
        protected void dataName(ObjFile obj) throws Exception {
            LexemeTokenWrapper wrapper = lexer.lookAheadK(1).get(0);
            if (wrapper.getToken().equals(Token.String)) {
                Optional<DataType> dataType = DataType.of(wrapper.lexeme);
                if (!dataType.isPresent()) {
                    obj.getDataSegment().addNmae(wrapper.getLexeme());
                    lexer.getNextToken();
                    dataName(obj);
                }
            }
        }
    
        protected void dataDefine(ObjFile obj) throws Exception {
            LOGGER.info("in data Define");
            dataName(obj);
            Token token = lexer.getNextToken();
            LOGGER.info("in data Define, token " + token);
            if (token.equals(Token.String)) {
                String lexem = lexer.getStatus().getCurrentLexeme();
                Optional<DataType> dataType = DataType.of(lexem);
                if (dataType.isPresent()) {
                    dataList(obj, dataType.get());
                } else {
                    throw new Exception("not find data type define in data segment at line " + lexer.getStatus().getLineIndex());
                }
            }
            LOGGER.info("exit data define");
        }
    
        protected void moreDataDefine(ObjFile obj) throws Exception {
            LexemeTokenWrapper wrapper = lexer.lookAheadK(1).get(0);
            if (wrapper.getToken().equals(Token.EndofContent)) {
                return;
            }
            LOGGER.info("wrapper token " + wrapper.getToken() + " lexeme (" + wrapper.getLexeme() + ")");
            if (wrapper.getToken().equals(Token.NewLine)) {
                match(Token.NewLine);
            }
            wrapper = lexer.lookAheadK(1).get(0);
            if (wrapper.getToken() != Token.DotString) {
                dataDefine(obj);
                moreDataDefine(obj);
            }
        }
    
        protected void dataList(ObjFile obj, DataType dataType) throws Exception {
            data(obj, dataType);
            moreData(obj, dataType);
    
        }
    
        protected void match(String str) throws Exception {
            lexer.getNextToken();
            if (!str.equals(lexer.getStatus().getCurrentLexeme())) {
                throw new Exception("Expected " + str + " at line " + lexer.getStatus().getLineIndex() +
                        ". However, we got " + lexer.getStatus().getCurrentLexeme());
            }
        }
    
        protected void match(Token t) throws Exception {
            Token token = lexer.getNextToken();
            if (!t.equals(token)) {
                throw new Exception("Expected token " + t + " at line " + lexer.getStatus().getLineIndex() +
                        ". However, we got " + token);
            }
        }
    
        /**
         * For now, only supports 3 kinds of data.
         * They are string, like '1dfdssasad'.
         * Number, like 123.
         * And dup, like dup 10 (123).
         * We should support $ and basic arithmetic operations here later.
         *
         * @param obj      the return object file
         * @param dataType see Enum dataType
         * @throws Exception exception
         */
        protected void data(ObjFile obj, DataType dataType) throws Exception {
            LexemeTokenWrapper wrapper = lexer.lookAheadK(1).get(0);
            if (wrapper.getToken().equals(Token.Quote)) {
                match(Token.Quote);
                match(Token.String);
                obj.getDataSegment().addData(lexer.getStatus().currentLexeme, dataType);
                match(Token.Quote);
            } else if (wrapper.getToken().equals(Token.String) && "dup".equals(wrapper.getLexeme())) {
                int currentLocation = obj.getDataSegment().getCurrentLocation();
                match(Token.String);
                match(Token.Number);
                int times = Integer.valueOf(lexer.getStatus().getCurrentLexeme());
                LOGGER.info("dup times " + times + " current location " + currentLocation);
                match(Token.LeftParent);
                dataList(obj, dataType);
                LOGGER.info("after parse data list location " + obj.getDataSegment().getCurrentLocation());
                List<Byte> tmpData = obj.getDataSegment().getPortionFrom(currentLocation);
                for (int i = 1; i < times; i++) {
                    obj.getDataSegment().addData(tmpData);
                }
                match(Token.RightParent);
            } else {
                BigInteger number = expr(obj);
                obj.getDataSegment().addData(number, dataType);
            }
        }
    
    
        protected BigInteger expr(ObjFile obj) throws Exception {
            BigInteger term = term(obj);
            return moreTerm(obj, term);
        }
    
        protected BigInteger term(ObjFile obj) throws Exception {
            BigInteger factor = factor(obj);
            return moreFactor(obj, factor);
        }
    
        //negative is not very good here
        protected BigInteger moreTerm(ObjFile obj, BigInteger left) throws Exception {
            LexemeTokenWrapper wrapper = lexer.lookAheadK(1).get(0);
            if (wrapper.getToken().equals(Token.Add)) {
                match(Token.Add);
                BigInteger right = term(obj);
                return moreFactor(obj, left.add(right));
            } else if (wrapper.getToken().equals(Token.Sub)) {
                match(Token.Sub);
                BigInteger right = term(obj);
                return moreFactor(obj, left.subtract(right));
            }
            return left;
        }
    
    
        protected BigInteger factor(ObjFile obj) throws Exception {
            LexemeTokenWrapper wrapper = lexer.lookAheadK(1).get(0);
            BigInteger number = null;
            if (wrapper.getToken().equals(Token.Sub)) {
                match(Token.Sub);
                return factor(obj).negate();
            } else if (wrapper.getToken().equals(Token.LeftParent)) {
                match(Token.LeftParent);
                number = expr(obj);
                match(Token.RightParent);
                return number;
            }
            wrapper = lexer.lookAheadK(1).get(0);
            if (wrapper.getToken().equals(Token.Number)) {
                match(Token.Number);
                number = new BigInteger(lexer.getStatus().currentLexeme);
            } else if (wrapper.getToken().equals(Token.String)) {
                match(Token.String);
                final String str = lexer.getStatus().getCurrentLexeme();
                if ("sizeof".equals(str)) {
                    match(Token.String);
                    Optional<DataType> type = DataType.of(lexer.getStatus().getCurrentLexeme());
                    if (!type.isPresent()) {
                        throw new Exception("not find data type define after sizeof operator at line " + lexer.getStatus().getLineIndex());
                    } else {
                        number = BigInteger.valueOf(type.get().getSize());
                    }
                } else {
                    int location = obj.getDataSegment().getLocationByName(str);
                    if (location == -1) {
                        throw new Exception("not find label " + str + " at line " + lexer.getStatus().getLineIndex());
                    } else {
                        number = BigInteger.valueOf(location);
                    }
                }
            } else if (wrapper.getToken().equals(Token.DollarSign)) {
                match(Token.DollarSign);
                int location = obj.getDataSegment().getCurrentLocation();
                number = BigInteger.valueOf(location);
            }
            if (number == null) {
                throw new Exception(" number is not defined at line " + lexer.getStatus().getLineIndex());
            }
            return number;
        }
    
    
        protected BigInteger moreFactor(ObjFile obj, BigInteger left) throws Exception {
            LexemeTokenWrapper wrapper = lexer.lookAheadK(1).get(0);
            if (wrapper.getToken().equals(Token.Mul)) {
                match(Token.Mul);
                BigInteger right = factor(obj);
                return moreFactor(obj, left.multiply(right));
            } else if (wrapper.getToken().equals(Token.Div)) {
                match(Token.Div);
                BigInteger right = factor(obj);
                return moreFactor(obj, left.divide(right));
            }
            return left;
        }
    
        public void moreData(ObjFile obj, DataType dataType) throws Exception {
            LexemeTokenWrapper wrapper = lexer.lookAheadK(1).get(0);
            if (wrapper.getToken().equals(Token.Comma)) {
                match(Token.Comma);
                data(obj, dataType);
                moreData(obj, dataType);
            }
        }
    
    
        protected void dataSegment(ObjFile obj) throws Exception {
            LexemeTokenWrapper wrapper = lexer.lookAheadK(1).get(0);
            if (wrapper.token.equals(Token.DotString) && ".data".equals(wrapper.getLexeme())) {
                lexer.getNextToken();
            }
            moreDataDefine(obj);
        }
    }
    \end{lstlisting}
    
    \item\textbf {Eight-Bit Register} \\
    \begin{lstlisting}
    
    package Instructions;
    
    import javax.swing.text.html.Option;
    import java.util.Map;
    import java.util.Optional;
    import java.util.function.Function;
    import java.util.stream.Collectors;
    import java.util.stream.Stream;
    
    /**
     * All registers contain 8 bits data.
     */
    public enum EightBitsRegister implements Register {
        AL("000"),
        CL("001"),
        DL("010"),
        BL("011"),
        AH("100"),
        CH("101"),
        DH("110"),
        BH("111");
        private final String registerCode;
    
        EightBitsRegister(String code) {
            registerCode = code;
        }
    
        @Override
        public RegisterLength getRegisterLength() {
            return RegisterLength.EIGHT;
        }
    
        @Override
        public String getRegisterName() {
            return name().toLowerCase();
        }
    
        @Override
        public String getRegisterCode() {
            return registerCode;
        }
    
        private static Map<String, Register> codeToRegister =
                Stream.of(values()).collect(Collectors.toMap(Register::getRegisterCode, Function.identity()));
    
        public static Optional<Register> of(String registerCode) {
            return Optional.ofNullable(codeToRegister.get(registerCode));
        }
    
        private static Map<String, Register> nameToRegister =
                Stream.of(values()).collect(Collectors.toMap(Register::getRegisterName, Function.identity()));
    
        public static Optional<Register> fromName(String name) {
            return Optional.ofNullable(nameToRegister.get(name.toLowerCase()));
        }
    }
    \end{lstlisting}
    \item\textbf {Operation Code} \\
    \begin{lstlisting}
    package Instructions;
    
    import common.BitSetUtils;
    
    import java.util.BitSet;
    import java.util.Map;
    import java.util.Optional;
    import java.util.function.Function;
    import java.util.stream.Collectors;
    import java.util.stream.Stream;
    
    /**
     * All Enum types inherit from Op interface.
     */
    public enum OpCode implements Op {
        ADD("000000"),
        ADC("000001"),
        SUB("000010"),
        SBB("000011"),
        MUL("000100"),
        DIV("000101"),
        MOV("000110"),
        PUSH("000111"),
        POP("001000"),
        CBW("001001"),
        CWD("001010"),
        BSWAP("001011"),
        AND("001100"),
        OR("001101"),
        XOR("001110"),
        NOT("001111"),
        SAL("010000"),
        SHL("010001"),
        SAR("010010"),
        SHR("010011"),
        CMPSB("010100"),
        CMPSW("010101"),
        CMPSD("010110"),
        LODSB("010111"),
        LODSW("011000"),
        LODSD("011001"),
        STOSB("011010"),
        STOSW("011011"),
        STOSD("011100"),
        REP("011101"),
        REPNE("011110"),
        JMP("011111"),
        JCC("100000"),
        JECXZ("100001"),
        CALL("100010"),
        RET("100011"),
        ENTER("100100"),
        LEAVE("100101"),
        LOOP("100110"),
        LOOPE("100111"),
        LOOPZ("101000"),
        LOOPNE("101001"),
        LOOPNZ("101010"),
        NOP("101011");
        private final String opcode;
    
        OpCode( String bits) {
            assert bits.length() == Op.SIZE;
            opcode = bits;
        }
    
        @Override
        public String getMemonic() {
            return name();
        }
    
        @Override
        public BitSet getBits() {
            BitSet bitSet = null;
            try{
                bitSet = BitSetUtils.fromString(opcode);
            }
            catch (Exception e){
                getOpLogger().info("catch exception " + e);
            }
            return bitSet;
        }
    
        @Override
        public  String getOpCode(){
            return opcode;
        }
    
        public static Optional<Op> of(final String opcode) {
            return Optional.ofNullable(bitsToOpCode.get(opcode));
        }
    
        private static final Map<String, Op> memToOpCode = Stream.of(OpCode.values())
                .collect(Collectors.toMap(Object::toString, Function.identity()));
    
        private static final Map<String, Op> bitsToOpCode = Stream.of(OpCode.values())
                .collect(Collectors.toMap(Op::getOpCode, Function.identity()));
    
        public static Optional<Op> fromMem(final String mem) {
            return Optional.ofNullable(memToOpCode.get(mem.toUpperCase()));
        }
    }
    \end{lstlisting}
    \end{enumerate}
    


\end{document}
